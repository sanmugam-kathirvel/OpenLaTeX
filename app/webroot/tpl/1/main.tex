\nonstopmode
\documentclass[11pt]{article}
\title{Instructions for C and C++ \\
  Spoken Tutorials in Ubuntu Linux OS}
\author{Ashwini Patil}
\date{26 July, 2013}
\topmargin -1in
\textheight 10.5in
\textwidth 6.5in
\oddsidemargin 0in
\evensidemargin 0in
\usepackage{graphicx,multicol,amsmath}
\newenvironment{enumcpt}{\begin{enumerate} \topsep 0pt \partopsep 0pt 
	\parsep 0pt
	\itemsep 0pt \leftmargin -1in \rightmargin 0pt
}{\end{enumerate}}

\pagestyle{empty}
\thispagestyle{empty}
\begin{document}
\begin{minipage}[t]{0.15\textwidth}
  \includegraphics[width=\linewidth]{3t-logo.pdf}
\end{minipage} \hfill
\begin{minipage}[t]{0.65\textwidth}
  \begin{center}
	\vspace{-0.7in}
	\Large
	Instruction Sheet for Gyan \\
	\large
	Spoken Tutorial Team \\
	IIT Bombay
  \end{center}
\end{minipage} \hfill
\begin{minipage}[t]{0.12\textwidth}

\end{minipage}

\begin{multicols}{2}

\section{The procedure to practise}
  \begin{enumcpt}
\item You have been given a set of spoken tutorials and files.
\item You will typically do one tutorial at a time.
\item You may listen to a spoken tutorial and reproduce all the steps shown in the video.
\item If you find it difficult to do the above, you may consider listening to the \emph{whole} tutorial once and then practise during the second hearing.
  \end{enumcpt}
 
  
  \section{C and C++}
  \begin{enumcpt}
  	\item Click on {\tt "Select FOSS Category"} drop-down and choose {\tt "C and C++"}.
	\item Click on {\tt "Select Language"} drop-down and choose the language (English, Hindi, Marathi ...) in which you wish to learn.
	\item Click on {\tt "Locate Tutorial"} button.
	\item You will see a list of tutorials based on your selection.
	\item Start with the first tutorial in the displayed list.
  \end{enumcpt}

\section{First tutorial: First C \\ Program}
 \begin{enumcpt}
  	\item Locate the topic {\tt "First C Program"}.
 	\item To view the tutorial, click on the video player icon on the right of the selected topic.
  	\item The {\tt Outline} of the tutorial and the {\tt Pre-requisite} will be visible on the right of the player.
  	\item The links for {\tt Code Files} and {\tt Assignment} will be available below the player.
  	\item Click on the player and view the tutorial.
	\item At {\tt 0:55 mins}, pause the video. 
	\item Here the video shows how to open the {\tt "Terminal"} in {\tt Linux OS}.

\vspace*{0.5in}

\subsection {Instructions to practise on Linux OS}
\begin{enumcpt}
	\item The tutorials are explained on the {\tt Linux OS}.
	\item It will be easy for the Linux users to follow as instructed in the tutorial.
\end{enumcpt}

\subsection {Instructions to practise on Windows OS}
\begin{enumcpt}
	\item On Windows, one has to use {\tt "Command Prompt"}.
	\item To open the  {\tt "Command Prompt"} on Windows, press the {\tt  "Windows" key and "R"  key} simultaneously on your keyboard.  It will open the {\tt "Run"} prompt.
	\item At the prompt, type {\tt "cmd"} and click on {\tt "Ok"}.
	\item This will open the {\tt "Command Prompt"}.			
\end{enumcpt}

\subsection {Common instructions for Assignments}
\begin{enumcpt}
  	\item At the prompt, type {\tt cd Desktop/} and press {\tt "Enter"}.
  	\item Now type {\tt mkdir name-rollno-c-cpp} and press {\tt "Enter"}.
  	\\ {\tt (Eg. mkdir Ashwini-1-c-cpp)}
  	\item This will create a folder with your {\tt "name"} and {\tt "rollno"} on the {\tt Desktop}.
  	\item Type {\tt cd name-rollno-C++} and press {\tt "Enter"}.
  	\\ {\tt (Eg. cd Ashwini-1-c-cpp)}
		 \item This will take you to that particular folder.
  	\item Give a unique name to the files you save in your folder, so as to recognize it next time. \\ {\tt $($Eg. "Practice-01-c"$)$}
\vspace*{1in}

  	\item Remember to save all your work in your directory.
  	\item This will ensure that your files don't get over-written by someone else.
  	\item Remember to save your work from time to time, instead of saving it at the end of the task.
  	\item Attempt all the given assignments as instructed in the tutorial.
\end{enumcpt}

\subsection {Common instructions to use Code files}
\begin{enumcpt}
		 \item Click on the link {\tt "Code files"} below the video player and save it in your folder.
	\item Extract the downloaded zip file.
	\item You will see all the code/source files used in the particular tutorial.
	\item Use these files as per the instructions given in the particular tutorial. \\
\end{enumcpt}

  \item Play-pause-practise the whole tutorial.
  \item Once the tutorial is complete, click on the back button on the browser (top-left corner left-arrow button).
  \item Now, choose the next tutorial and follow all the above instructions, till you complete all the tutorials in the series.
\end{enumcpt}

\end{multicols}
\end{document}